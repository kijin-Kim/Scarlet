\chapter{Test\+Page}
\hypertarget{_test_page}{}\label{_test_page}\index{TestPage@{TestPage}}
첫 문장에는 동사를 사용해 메서드의 역할을 설명하십시오. \textquotesingle{}무엇을 반환한다\textquotesingle{}는 지양하십시오.

줄바꿈한 다음 아래와 같은 정보가 있으면 기술하십시오.
\begin{DoxyItemize}
\item 이 메서드를 호출하기 전후에 해야 하는 작업이 있다면 기술하십시오.
\item 특정한 상황에서 이 메서드를 사용하지 말아야 한다면 이유를 설명하십시오. {\itshape 참고} 영어로 쓴다면 첫 글자는 대문자로 쓰십시오.
\end{DoxyItemize}


\begin{DoxyParams}{Parameters}
{\em base} & 어떤 의미의 파라미터인지 쓰십시오. boolean 값이라면 언제 true이고 언제 false 인지 쓰십시오. 숫자라면 범위가 있는지 쓰십시오. enum이라면 그 enum 항목을 링크하십시오. 허락되지 않은 값을 전달했을 때 무슨 일이 발생하는지 쓰십시오. \\
\hline
\end{DoxyParams}
\begin{DoxyReturn}{Returns}
무엇을 반환하는지 명사로 쓰십시오. boolean 값이라면 언제 true이고 언제 false 인지 쓰십시오. 숫자라면 범위가 있는지 쓰십시오. enum이라면 그 enum 항목을 링크하십시오. 문제가 발생했을 때 일반 상황과 다른 의미의 값을 반환한다면 기술하십시오. (예\+: -\/1을 반환하는 경우) 
\end{DoxyReturn}
\begin{DoxySeeAlso}{See also}
유사한 기능을 하는 메서드가 있다면 기술하십시오. 어떤 스펙을 구현한 것이라면 그 스펙의 이름 혹은 링크를 기술하십시오. 
\end{DoxySeeAlso}
